% Custom preamble based on --type=pdfcomment --driver xetex

% Mark up added text with a blue underline, and removed text with a red strikethrough.
% Choose blue/red instead of green/red for accessibility (red-green colorblindness modes constitute the most common forms).
\RequirePackage{pdfcomment}
% pdfmarkupcomment is pretty cool but has some issues:
% 1. Doesn't work in xltabular environment (just doesn't show anything)
% 2. Not every PDF viewer displays the comment thread (e.g., Chrome shows only the markup)
% 3. You get one comment per line:
%    https://tex.stackexchange.com/questions/474947/highlighting-text-that-spans-multiple-lines-with-pdfmarkupcomment-from-pdfcomme
% \providecommand{\DIFadd}[1]{\pdfmarkupcomment[author=Inserted,markup=Underline,color=blue]{#1}{#1}}
% \providecommand{\DIFdel}[1]{\pdfmarkupcomment[author=Deleted,markup=StrikeOut,color=red]{#1}{#1}}
% hspaces are here to ensure that on all PDF readers, the comment indicator does not cover up the text.
\providecommand{\DIFadd}[1]{{\color{blue} \uline{#1}}}
\providecommand{\DIFdel}[1]{{\color{red} \sout{#1}}}
% Basic add/del/mod environments.
\providecommand{\DIFaddbegin}{\pdfcomment[author=Inserted,icon=Note,color=blue]{Text} \hspace{16pt}}
\providecommand{\DIFaddend}{}
\providecommand{\DIFdelbegin}{\pdfcomment[author=Deleted,icon=Note,color=red]{Text} \hspace{16pt}}
\providecommand{\DIFdelend}{}
\providecommand{\DIFmodbegin}{}
\providecommand{\DIFmodend}{}

% For floats, we use the graphics macros below.
\providecommand{\DIFaddFL}[1]{#1}
\providecommand{\DIFdelFL}[1]{#1}
\providecommand{\DIFaddbeginFL}{\pdfcomment[author=Inserted,icon=Note,color=blue]{Image}}
\providecommand{\DIFaddendFL}{}
\providecommand{\DIFdelbeginFL}{\pdfcomment[author=Deleted,icon=Note,color=red]{Image}}
\providecommand{\DIFdelendFL}{}

% % Latexdiff by default wants to shrink deleted figures. Let's try not doing that,
% % but if we want to change this later, here's what was here once.
% \newcommand{\DIFscaledelfig}{1.0}
% \RequirePackage{settobox}
% \RequirePackage{letltxmacro}
% \newsavebox{\DIFdelgraphicsbox}
% \newlength{\DIFdelgraphicswidth}
% \newlength{\DIFdelgraphicsheight}
% % store original definition of \includegraphics %DIF PREAMBLE
% \LetLtxMacro{\DIFOincludegraphics}{\includegraphics} %DIF PREAMBLE
% \newcommand{\DIFaddincludegraphics}[2][]{{\color{blue}\fbox{\DIFOincludegraphics[#1]{#2}}}} %DIF PREAMBLE
% \newcommand{\DIFdelincludegraphics}[2][]{% %DIF PREAMBLE
% \sbox{\DIFdelgraphicsbox}{\DIFOincludegraphics[#1]{#2}}% %DIF PREAMBLE
% \settoboxwidth{\DIFdelgraphicswidth}{\DIFdelgraphicsbox} %DIF PREAMBLE
% \settoboxtotalheight{\DIFdelgraphicsheight}{\DIFdelgraphicsbox} %DIF PREAMBLE
% \scalebox{\DIFscaledelfig}{% %DIF PREAMBLE
% \parbox[b]{\DIFdelgraphicswidth}{\usebox{\DIFdelgraphicsbox}\\[-\baselineskip] \rule{\DIFdelgraphicswidth}{0em}}\llap{\resizebox{\DIFdelgraphicswidth}{\DIFdelgraphicsheight}{% %DIF PREAMBLE
% \setlength{\unitlength}{\DIFdelgraphicswidth}% %DIF PREAMBLE
% \begin{picture}(1,1)% %DIF PREAMBLE
% \thicklines\linethickness{2pt} %DIF PREAMBLE
% {\color[rgb]{1,0,0}\put(0,0){\framebox(1,1){}}}% %DIF PREAMBLE
% {\color[rgb]{1,0,0}\put(0,0){\line( 1,1){1}}}% %DIF PREAMBLE
% {\color[rgb]{1,0,0}\put(0,1){\line(1,-1){1}}}% %DIF PREAMBLE
% \end{picture}% %DIF PREAMBLE
% }\hspace*{3pt}}} %DIF PREAMBLE
% } %DIF PREAMBLE

% Add a PDF comment and put a box around the added and removed graphics.
% store original definition of \includegraphics so we can wrap it.
\LetLtxMacro{\DIFOincludegraphics}{\includegraphics}
\newcommand{\DIFaddincludegraphics}[2][]{
	{\color{blue}\fbox{\DIFOincludegraphics[#1]{#2}}}
} %DIF PREAMBLE
\newcommand{\DIFdelincludegraphics}[2][]{
	{\color{red}\fbox{\DIFOincludegraphics[#1]{#2}}}
} %DIF PREAMBLE

\LetLtxMacro{\DIFOaddbegin}{\DIFaddbegin} %DIF PREAMBLE
\LetLtxMacro{\DIFOaddend}{\DIFaddend} %DIF PREAMBLE
\LetLtxMacro{\DIFOdelbegin}{\DIFdelbegin} %DIF PREAMBLE
\LetLtxMacro{\DIFOdelend}{\DIFdelend} %DIF PREAMBLE
\DeclareRobustCommand{\DIFaddbegin}{\DIFOaddbegin \let\includegraphics\DIFaddincludegraphics} %DIF PREAMBLE
\DeclareRobustCommand{\DIFaddend}{\DIFOaddend \let\includegraphics\DIFOincludegraphics} %DIF PREAMBLE
\DeclareRobustCommand{\DIFdelbegin}{\DIFOdelbegin \let\includegraphics\DIFdelincludegraphics} %DIF PREAMBLE
\DeclareRobustCommand{\DIFdelend}{\DIFOaddend \let\includegraphics\DIFOincludegraphics} %DIF PREAMBLE
\LetLtxMacro{\DIFOaddbeginFL}{\DIFaddbeginFL} %DIF PREAMBLE
\LetLtxMacro{\DIFOaddendFL}{\DIFaddendFL} %DIF PREAMBLE
\LetLtxMacro{\DIFOdelbeginFL}{\DIFdelbeginFL} %DIF PREAMBLE
\LetLtxMacro{\DIFOdelendFL}{\DIFdelendFL} %DIF PREAMBLE
\DeclareRobustCommand{\DIFaddbeginFL}{\DIFOaddbeginFL \let\includegraphics\DIFaddincludegraphics} %DIF PREAMBLE
\DeclareRobustCommand{\DIFaddendFL}{\DIFOaddendFL \let\includegraphics\DIFOincludegraphics} %DIF PREAMBLE
\DeclareRobustCommand{\DIFdelbeginFL}{\DIFOdelbeginFL \let\includegraphics\DIFdelincludegraphics} %DIF PREAMBLE
\DeclareRobustCommand{\DIFdelendFL}{\DIFOaddendFL \let\includegraphics\DIFOincludegraphics} %DIF PREAMBLE

% Use a code-diff style for diffs in code
\RequirePackage{listings}
\lstdefinelanguage{DIFcode}{
  moredelim=[il][\color{red}]{\%DIF\ <},
  moredelim=[il][\color{blue}]{\%DIF\ >}
}
\lstdefinestyle{DIFverbatimstyle}{
	basicstyle=\linespread{1.0}\small\ttfamily{},
	language=DIFcode,
	basicstyle=\ttfamily,
	columns=fullflexible,
	keepspaces=true
}
\lstnewenvironment{DIFverbatim}{
	\pdfcomment[author=Changed,color=blue,icon=Note]{- Deleted, + Inserted}
	\lstset{style=DIFverbatimstyle}
}{}
\lstnewenvironment{DIFverbatim*}{
	\pdfcomment[author=Changed,color=blue,icon=Note]{- Deleted, + Inserted}
	\lstset{style=DIFverbatimstyle,showspaces=true}
}{}
\lstset{extendedchars=\true,inputencoding=utf8}

% N.B. a sed script in build.sh may look for the following line. Use caution when changing it.
% End Custom TCG extension for latexdiff
